\tolerance=1
\emergencystretch=\maxdimen
\hyphenpenalty=10000
\hbadness=10000

\chapter{Introduction}
\label{chap: into}
Mechatronics is an engineering field that comprises of mainly mechanical, electrical and computer science domains \cite{neumann_mechatronic_nodate}. Mechatronic product development means the independent development of mechanical, electrical and software parts that are later united to constitute the entire system. The process of development of such a system has numerous
challenges \cite{bitzer2016product} \cite{alvarez_cabrera_towards_2010}. These challenges are mainly due to the presence of a wide range of disciplines and stakeholders. To overcome these challenges and to promote collaborative, simultaneous engineering, Mcharek et al. propose a useful framework called ”Knowledge Based Engineering” in \cite{mcharek_knowledge_2018}. This helps the engineers in one of the main tasks — decision making \cite{walden2015}\cite{irshorn2007}. The study of mechatronics can be split into a variety of other focus areas like sensors, actuators, system modeling, logic systems etc. It involves complex decision making throughout the life cycle of product development process\cite{hamida_towards_nodate}. Therefore, mechatronics product development happens to be an extremely interactive domain. Due to this multi-disciplinary nature, the team members involved in mechatronics system design also come from multiple fields of engineering. This demands a high level of collaboration among the team members for a successful system design or product development \cite{neumann_mechatronic_nodate}. Collaboration tools become very significant in such a team for all it’s members to be aligned. Tools like Slack \cite{lin_why_2016} and Microsoft teams \cite{hubbard_mastering_2018} are extremely popular in the developer community. Additionally, such tools also aid the exchange of knowledge, information, updates and decisions about the development during the process. Furthermore, collaboration tools these days are highly sophisticated with numerous useful bots\cite{lebeuf_defining_2019}. Bots are nothing but an interactive interface for users to avail software services. They can also be termed as artificially intelligent, capable of comprehending commands from users or sometimes smart enough to carry out automated tasks. These bots are said to be highly convenient to carry out regular tasks. Conversational bots these days are so engaging and intelligent that it is often difficult to spot differences between a human and a bot\cite{muresan_chats_2019}. There are also software bots  that aid a variety of tasks related to code development, code maintenance and other productivity tasks through simple commands. Further, Carlene et al. highlights that these chat bots help reduce the collaboration friction among the software developers\cite{lebeuf_how_nodate}. So far, some of the fundamental responsibilities of a mechatronics team such as design, decision making and collaboration through the product development process has been discussed. Going ahead, another salient task is highlighted --- documentation. There are numerous benefits attributed to the documentation activity. For instance, up-to-date knowledge is recorded and as a result, all the learning and know-hows is conserved. The documentation also serves as a road map for the further developmental activities. Lack of documentation leads to lack of maintenance, lack of complete, updated systems and lack of traceability of the crucial design decisions. Therefore, Heesch et al.in \cite{van_heesch_does_2013} discusses why decision documentation can be valuable for the junior engineers.

In this master thesis, the mechatronics systems/mechatronic product development process is being studied, different types of challenges in the product development process is assessed, with that knowledge relevant problems pertaining to documentation of design decisions is analyzed, how collaboration tools can act as a beneficial medium to gather key design decisions is explored and finally a bot that can help transform that data as a document for future reference is prototyped. To begin with, there will be initial screening of some of the existing bots to see if they can be adopted to help achieve the goal of this thesis in the context of documenting design decisions.

\section{Problem description}
\label{pd}
Mechatronics being a diverse team raises certain concerns in the course of product development life cycle \cite{bitzer2016product}\cite{neumann_mechatronic_nodate}. A few pertinent issues are studied in this masters thesis. Based on the literature review, there are two most important issues that needs to be addressed in a typical mechatronics team. These two are separate issues and are yet inter-dependent. The first problem causes the second problem below. So, problem number 2 is the major issue of concern.
\begin{enumerate}
\item  Among the multiple domains within a team, there is room for improvement to exchange knowledge with respect to terminologies, functional requirements, design parameters and developmental processes. Weak knowledge management can lead to confusions within team members and hamper the design and development. There needs to be an organized method for the required knowledge sharing task. Furthermore, as mentioned above, mechatronics product development implies intricate details about design, architecture, implementation decisions and once again documenting(knowledge management) such important decisions become significant \cite{neumann_mechatronic_nodate}. Documentation in turn promote knowledge exchange activity.

\item The major issue is lack of care for explicit documentation, particularly decision documentation. From the problem description number 1, the inference is that poor documentation is a consequence of poor knowledge management. Therefore, the focus here is to study the decision making challenges\cite{hamida_towards_nodate} and going further, the need to document all the decisions. Decision documentation is a process of capturing vital information in a readable format, often required for future references. The authors in \cite{manteuffel_decision_2016} highlight the fact that documentation of architectural decisions in the industry are usually neglected. The cost of poor or no documentation in the field of systems engineering can be well comprehended from \cite{kasser_improving_1995}. Therefore, there is a clear purpose for documentation in mechatronics product development. Since collaboration tools are a critical medium for co-ordination among mechatronics team members, most of such decisions lie inside the collaboration tools. They are often present in the chat logs or sometimes even lost during the online meetings. How to differentiate between productive and unproductive material in chats/group chats \cite{zhang_making_2018}? How do we extract any pivotal design decision that was made on a collaboration tool? For instance, on call whiteboards are increasingly getting popular lately, how does the team manage to capture any architectural/design decision that could be useful at a later stage \cite{gilson2019natural}? Besides, going through long threads of chats for any vital data or decision will be a tedious task. How do we make this process hassle-free? Henceforth, there is need for motivation to manage knowledge and documentation in such a team and also have a more structured mechanism to tackle the above issues. There have been several tools as part of modeling platforms in order to record the design decisions\cite{manteuffel_decision_2016} where as there are no means yet to secure the design decisions that are present in collaboration tools.
\end{enumerate}

\section{Objectives}
\label{obj}
The primary objective of this master thesis is to aid employee motivation specifically with a focus to uphold design decisions documentation. The ultimate objective is to prototype a software bot that promotes employee motivation in the same focus area. Lately, software bots are useful in numerous fields like engineering\cite{storey_botse_nodate}, development \cite{abdellatif_msrbot_2020} and HR \cite{mohan2019chat}. This type of employee dependence on such software bots bespeaks that the bots are generally helpful in motivating employees in one way or the other in order to reach their milestones. Research carried out by Suri et al. \cite{oshri_software_2017} say that there is usually a business case developed for the usage of software bots in organizations. The research summarizes the factors that are critically important for development of such a business case in which employee motivation constitutes 71\%. Thus, it can be deduced that software bots boost employee motivation in general. Therefore, in this thesis, employee motivation to documentation is achieved through design and development of a software bot but, considering the scope of the thesis, motivation to document design decisions is the main focus. The objectives of the bot are intended to drive the design of a software bot and help decide on the features to be implemented. The software bot will have one major objective and 2 optional objectives as follows. These in turn reflects the features of the bot that is discussed in the section \ref{si}. 
\begin{enumerate}
\item This is the major objective and holds a higher value in priority of execution. The goal is to find an optimal solution to trace the design decisions present in collaboration tools. The employees use these tools for regular communication and co-ordination \cite{lin_why_2016}. Many prominent decisions would be made over these tools in group chats. There could be design decisions, organizational decisions, management related decisions. They can all be present jointly in a single chat. Later on, it is not a simple task to go through them to access any desired message that indicated a design related decision \cite{zhang_making_2018}. Hence, the aim is to mark the relevant messages using a bot in a chat during desired conversation so as to make them handy later when required \cite{gilson2019natural}.
\end{enumerate}

Next up are the two optional objectives that hold a lesser value in priority of execution. The execution of these depend on the time remaining after a successful execution of the major objective explained above. Out of the below two objectives, the former one solves the possible inconsistency in knowledge management that can occur in an inter-disciplinary team. The latter one talks about a more general employee motivation with respect to mutual acknowledgment in a team like mechatronics.   
\begin{enumerate}
\item This optional objective is to improve knowledge exchange spanning across multiple teams. Lack of knowledge exchange hampers the collaboration thus leading to poor motivation among the employees \cite{neumann_mechatronic_nodate}. There are knowledge about design models from different disciplines and it is challenging to evaluate all of them in parallel \cite{alvarez_cabrera_towards_2010}. Huge amounts of complex information or knowledge sharing would help the team to align better during product development process. This approach will also help the employees to obtain a holistic perspective on a product
development which in turn adds up to employee motivation. Nevertheless, a lot of collaboration tools \cite{hubbard_mastering_2018} are available for this purpose but the idea here is to use these tools to further exchange information in an effective and an efficient manner with the help of a bot.

\item The next optional objective is concerning employee recognition. It is said that employee motivation enhances both an individual’s performance as well as an organization’s performance \cite{montani_employee_2020}. These recognition could be managerial-based recognition or coworker-based recognition. This helps employees to develop a sense of positive competition among themselves which directly influences the company’s growth. In this case, employee recognition takes place in collaboration tools and there are already a couple of software bots like HeyTaco\footnote{https://www.heytaco.chat/} and Achievers\footnote{https://www.achievers.com/gb/} that help to encourage employees to complete their assignments. Once again, the purpose of this objective is to encourage positivity and motivation among employees using a bot.
\end{enumerate}
 

\section{Solution idea}
\label{si}

The features of the motivational bot are listed in this section but like mentioned above, there is a \textbf{\textit{key}} feature that is prioritized and is implemented first. Although, a software bot that is capable of rendering all three features is desired for a mechatronics team, the preference is given to the major objective that drives the \textit{\textbf{key}} feature as follows.

The proposed solution in this work is to device an intelligent bot that supports employee motivation mainly in the process of documentation in a team like mechatronics. Documentation is even further narrowed down to design decision documentation. The initial step for this is to firstly evaluate the existing bots, i.e., to carry out a research to see if there are bots already capable of fulfilling the major objective stated in \ref{obj}. If there are bots that are found to suit the requirements, the work will be towards adapting it’s functionalities to prototype a bot that is exclusive to mechatronics domain. If there are no such existing bots, design and prototype of an appropriate bot that solves the major issue pointed out in \ref{pd} will follow. This bot will be implemented to be added in Microsoft Teams.


\begin{itemize}
\item The \textbf{\textit{key}} feature of this bot is to track all the important design decisions that are present in an unorganized fashion inside a channel of Microsoft Teams. The bot should provide suitable means to differentiate the messages between general discussions and design related decisions in the group chats and present either a dashboard or a PDF file in a readable format. The dashboard and the file would display a summary of all the discussed design decisions, decision makers, date of decision and type of decisions. The user should be allowed to save the file to their local system. The default file name can be the combination of the channel name and a term specific to mechatronics design phase so that, it reflects the name of a popular mechatronics terminology that is universally perceived. This way, it is easier for a user to refer to it at a later stage.
\end{itemize}

The below features may or may not be implemented in this work. If the \textbf{\textit{key}} feature is successful well before the deadline, the following optional features will be implemented otherwise, this can be considered as part of future work. However, it is good to discuss ideas for now or later.
\begin{itemize}
\item This feature is to tackle the issue of lack of knowledge exchange in a mechatronics team. The bot can act as a translator for terminologies, knowledge base in terms of data, information, requirement parameters and development processes thus helping
the knowledge flow across the inter-disciplinary teams. This may also be termed as a center for any FAQs. For instance — A user from a certain domain A should be able to interact with the bot by querying, through simple commands and receive the desired information pertaining to a different domain B.

\item The next feature is to motivate employees by acknowledging their work and efforts in any project. The bot is designed to appreciate the team members who have worked to successfully attempt an innovative design strategy with respect to architecture and design or members who made substantial git commits and have closed high priority JIRA\footnote{https://www.atlassian.com/software/jira} stories/issues/tasks with respect to development or members who discovered a path-breaking strategy with respect to project management. The bot grants reward points for every task completed in their own focus area. For instance --- in the area of software development, the reward points depend on tasks completed versus bugs returned before a sprint is over. After a certain number of points have been achieved, the employee could be made eligible for a more valuable reward. Certain gamification ideas can be used in this case, the messages can appear like in a video game. For example, after every 5 tasks successfully closed(without any bugs returned), the bot could posts a motivating message like — Hurray! Level 1 completed! It is now time for you to accomplish next challenging level and so on.
\end{itemize} 


\textit{Note: The above features form the proposed solution and if there are any modifications during implementation in the near future, appropriate reasons and arguments to any changes shall be formulated accordingly.}

\section{Thesis outline} This section describes how rest of the thesis report is organized.

Chapter \ref{chap: bg} gives a brief overview of the fundamental concepts required to understand further chapters in the report. This helps the readers to know a bit of history of mechatronics, process of decision making and  general idea about collaboration tools and software bots.

Chapter \ref{chap: ls} summarizes the literature survey that was carried out during the thesis tenure. This survey describes the avenue taken to understand the discipline of mechatronics, it's product development, decision making procedures, challenges in mechatronics domain and best practices. The survey then takes a shift to understand collaboration tools, it's advantages and disadvantages and finally the study of software bots and how they can be made use to complete tasks in an easier way.

Chapter \ref{chap: eoeb} describes the research task of this thesis topic i.e. Evaluating the existing bots to verify if there is already a functionality in place to tackle the problem description. If yes, how should it be upgraded to develop a better version. If not, then the development of a bot pertaining to the objective follows.

Chapter \ref{chap: dt} shows how the bot is designed. The architecture, workflow and technologies used for the design are discussed in detail. Chapter \ref{chap: id} talks about the implementation aspects of the bot. Sequence diagrams, code snippets and the screenshots of the working bot are presented to help the readers understand the technicalities of the bot.

Chapter \ref{chap: vod} is the validation of the developed bot. It proves the bot that has been developed fulfills the requirements and objectives stated. 

Chapter \ref{chap: cafe} concludes the report by suggesting some of the pointers for future enhancements.
