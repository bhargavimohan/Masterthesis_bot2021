\chapter{Literature survey}
\label{chap: ls}

There are two sets of literature survey in this thesis. One is the super-set concerning the chief topics and is outlined in this particular section. Another is the sub-set which is more specific to one of the chief topics and is discussed in Chapter \ref{chap: eoeb}

The aim of this section is to explain in what order the scientific papers were studied. The outcomes of referring the following papers helped to re-define the problem description and comprehend what is lacking and what are the requirements clearly. It also aided to emphasize the apparent objectives and played a great role in establishing a holistic view of the entire thesis topic.

First, the plan was to simply understand how product development takes place in a mechatronics environment and the authors in \cite{neumann_mechatronic_nodate} explore the strengths and weaknesses of Mechatronic Product Development(MPD). Due to the complexity of the multi-disciplinary domains, all the challenges with respect to the terminologies are exhibited. It presents the terminological clarifications that is required in MPD. One other main point here is knowledge sharing among the different domains. The paper highlights about benefits of this cross platform product development in terms of cost and features. It also talks about the drawbacks of the same in terms of complexity, development process and structure of MDP. Subsequently, the industry process of product development can be well understood in \cite{bitzer2016product}. The paper explains in detail about Product Lifecycle Management (PLM), state-of-the-art challenges faced by PLM, the authors have proposed an approach for a better methodology and application of PLM process. The authors quote an industrial example, of a client case and hence have helped evolve PLM considering the current challenges.
The authors of \cite{alvarez_cabrera_towards_2010} in contrast reviews challenges in design of mechatronics systems. These challenges are mainly in the phase of integration of designs from different domains. The authors stress the need to consider the complex inter dependencies between subsystems. Further studies regarding the challenges of development in mechatronics led to \cite{thramboulidis_challenges_2008}. The fact that the already existing mechanical and electrical systems are outdated. These legacy systems have to be replaced by updated systems where the functionality is taken care by software implementation. The authors also discuss the challenges in the development of mechatronic systems and a mechatronic component is also proposed. The main aim of this component is to aid those identified challenges in terms of integration, complexity, size. Further more, the survey journey had to associate the challenges of mechatronics to the next chief topic i.e. collaboration. The paper \cite{mcharek_knowledge_2018} serves as the link and rightly explains the challenges and the need for collaboration. This paper briefs the importance of collaboration and knowledge sharing among different disciplines. Knowledge based engineering method is presented which is adapted to mechatronic design. This method gives a way to help engineers to achieve collaboration therefore reducing design time and cost.

The survey now focuses on collaboration tools. Microsoft Teams and Slack tools were given most attention. Hence the book \cite{hubbard_mastering_2018} is regarding everything about Microsoft Teams and its features. It includes explanation about its characteristics and how it can be operated. It includes some background study that is its history with the application Skype\footnote{https://www.skype.com/en/}. The book gives all the essentials of this collaboration tool like creating teams, channels, scheduling meetings, automation of some processes with the help of teams, third party add-on like bots etc. Similarly, the popularity of Slack tool is well detailed in \cite{lin_why_2016}. It states how developers have been using Slack as their effective and efficient tool for collaboration during software development. The extensive use of bots, and the possible integration of other services is a major aspect why slack is so renowned in the developer community. This study is to mainly understand the significance of Slack that supports various processes of software engineering. The scientific work in \cite{zhang_making_2018} identifies how can common messages be distinguished by noteworthy messages in collaboration tools. This paper talks about ways to distinguish important and unimportant messages in chat logs. They emphasize the need for group chat analysis towards making it more structured. The tool ``Tilda" helps in summarizing chats for the users using simple commands. The authors evaluate the tool against google docs\footnote{https://docs.google.com/document/u/0/} and different forms of notes taking and discuss the efficiency of the tool implemented. Hence, it is clear that the distinction and documentation of noteworthy messages, or a display of summary of all those messages in the collaboration tools would be a coveted solution for the problem description in the thesis. The type of messages that needs to be captured for later use is design decisions as per the requirement. This led to the next chief literature survey topic i.e. decision making process to understand the internal details of the process.

Initially, the study of decision methods had to be made. So, two well known books were referred to get a thorough and comprehensive understanding of decision making procedures in mechatronics. In \cite{walden2015} the decision management has been explained in detail. It states decision situations in every phase of the lifecycle. It describes the key elements of a decision management process like input and output variables and the process activities. The process activities include defining decision management strategy, listing the differences when an alternate decision is chosen, analyzing and managing the decisions. It also talks about alternative decisions through deterministic analysis and improving alternatives. Likewise, \cite{irshorn2007} gives an overview of decision analysis process. It tabulates different review process in different phases of the product development. It also lists the purpose, timing and results of the review process.
This paper highlights the fact that these review processes are helpful in the evaluation of technicalities , expenses and alternatives. It summarizes the evaluation methods and tools for the purpose of assessing a decision that was taken. Evaluation methods such as simulations, weighted trade off matrices, surveys, user review and comment and a few more can be used. In addition, the study of tools or systems that support the decision systems was interesting hence papers that give a good synopsis about decision support systems were examined. The work in \cite{chai_zhengmeng_brief_2011} says the Decision Support Systems (DSS) have been well investigated over 40 years and further talks about the development of DSS applications. The authors review the progress of DSS research and examine Model-driven DSS, Data-driven DSS, Group Communication-driven DSS, Document-driven DSS, Knowledge-Driven DSS, Web-based DSS and finally explain the future of DSS. In much the same way, \cite{hersh_sustainable_1999} demonstrates the role of DSS in sustainable decision making. General principles of sustainable decision making is presented. Also, some examples where DSS can be applied to sustainable decision making are quoted. Authors have also discussed approaches to decision making and decision support systems. Finally, the academic work proposed in \cite{hamida_towards_nodate} signifies the need of inter-dependent decision making in the design phase of complex systems. The authors examine this situation by conducting interviews and work-shop with system architects. This helped them understand what decision support tools are required in the system architecture process. The gathered data from architects later helped authors to develop something called decision support Framework. This framework has 7 different decision domains related to system architecture life cycle. At this point, it became clear that a system/device/mechanism to support design and architectural decisions in a mechatronics field is the core requisite. Since communication is utmost important among all disciplines, the system/device/mechanism to support design decisions is best to be equipped inside collaboration tool in the form of a BOT.  

Finally, the software bots needed to be studied. The authors in \cite{lebeuf_software_2018} depicts the advancement of software bots and their extensive usage. This paper gives an overview of how software development and bots go hand-in-hand. The paper gives a short description about creating and hosting bots. This paper also classifies the bots as collaboration bots, productivity bots, documentation bots etc. The paper lists some common bot platforms along with other bot technologies. Next, the authors of \cite{lebeuf_defining_2019} have presented a taxonomy of software bots. It is based on environment dimension, intrinsic dimension and interaction dimension. This makes it very clear about the properties, behaviors and the environment where the bots are being operated and designed. Also, \cite{lebeuf_how_nodate} examines how chat bots help reduce collaboration friction points in software development. The authors classify and present 3 friction points.  The authors present a sequence of research questions that forms a gateway to future discussions on bots that can assist team collaboration. Besides, \cite{muresan_chats_2019} is a study that is specific to one bot - Replika. The main aim of this experiment was to focus on the anthropomorphic view during the interaction. After this experiment, the participants were asked how could Replika be more engaging. Lastly, a seminar paper \cite{storey_botse_nodate} in which the authors brought together researchers and practitioners with diverse backgrounds. There were also questions raised on how to effectively design and use bots. There are definitions and outlook of bots, level of acceptance of bots by the developers, use cases of bots, inclusion of diverse community with the help of bots, bots that promote collaboration.

Some more specific literature review on bots to follow in Chapter \ref{chap: eoeb}








 