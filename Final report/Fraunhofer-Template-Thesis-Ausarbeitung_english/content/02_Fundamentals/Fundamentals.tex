\chapter{Background}
\label{chap: bg}

\section{Mechatronics}
The origin of Mechatronics engineering dates back to the end of 1960. The term was coined by a Japanese engineer "\textbf{Tetsura Mori}"\cite{dixit2017history}. It is only during the 1980's that the domain acquired distinction and is now well accustomed. The advantages of the mechatronics systems has obtained wide acceptance all over the globe. This domain has over 10 technical sub-domains within itself, in the area of its R\&D \cite{brighthubengineering_2010}. Therefore, the study of mechatronics systems is an amalgamation of various other technological and engineering subjects. It is serving as a gateway to massive scientific and technical studies. One notable observation is that the study of mechatronics is not only the combination of other engineering domains but also includes non-engineering domains \cite{bradley_2004}\cite{habib_mechatronics_2006}. Therefore, it recognizes the possibility of advanced ideation and conception.

The interdisciplinary nature of mechatronics is an essential viewpoint. Studies show that the technological advancements can not be distinguished as per the formal disciplines. The next-gen systems could probably be formulated  from the inter-working of multiple unrelated disciplines of operations in science and technology\cite{wikander_science_2001}. This science paradigm is also considered to have a philosophical side that motivates creative thinking, initiation of modern designs and modeling smart systems. The objective here is to explore advanced technological prospects by merging diverse domains together in order to produce new, intelligent techniques and systems approaches. The end product is manifested in a way that it is characterized with topnotch performance, accuracy, dependability, resilience and reasonable qualities. On the whole, the synergy in mechatronics systems design and development empowers great efficiency in terms of improved functionality, high reliability, better quality, effective scalability. Additionally, it is heading towards shortening developmental life cycle time and managing economic friendly methodologies in its evolution. Hence helping to provide simpler solutions to complex problems\cite{habib_mechatronics_2007}.

\section{Collaboration tools}
Collaboration in this context is defined as the activity of discussing task related topics among different members of a team. Tool in this context is a software application that assists a particular task\cite{lomas_collaboration_nodate}. Thus, collaboration tool supports communication on different levels among members.  Some of the categories of collaboration tools are version-control systems, ticketing systems, build tools, knowledge tools and communication tools.The main focus in the thesis is communication tools.  Some of the well known communication tools are emails, forum websites or software applications like Skype, MS teams, slack, Google Hangouts Meet etc. 

Communication among members of any team is made simple and easy by the use of the online tools. The team members can be situated in any part of the world to use collaborative tools irrespective of the geographical limitation. They can all come together to discuss their topic of interest through the collaboration tools. Fundamentally, the nature of a combined team of mechanical, electrical, software, in other words mechatronics, is that their foremost task is co-ordination among the team members. Collaboration tools act as enablers for them to discuss problems, solutions, decisions and ideas among other team members. Furthermore, they also aid knowledge sharing, information and project management\cite{lomas_collaboration_nodate}. In this thesis, slack and MS teams are the two main tools that is inspected. More information on these tools are discussed in the following sections.
 

\section{Decision making in mechatronics} The process of a good decision making is crucial in a team like mechatronics because it is the one that decides how other technical processes as well as management processes should be carried out. A decision, especially a design decision should be carefully examined to study its repercussions that it could have on the system later.  The design decisions should be made in the appropriate time because the premature design decisions could have serious impacts on system, cost and throughout the entire life cycle. Further, there will be decision gates defined that makes sure to team members that all the earlier tasks are up-to-date and are accomplished before getting along to the next set of tasks. Decision gates are also viewed as ``milestones" or ``reviews". The incoming and outgoing criteria are designated for every gate during the project management timeline. It is during this time in the project life cycle, a task is validated and approved by the project head\cite{walden2015}. 

The design decisions are crucial in Architecture definition process, Design/concept definition process as well as Implementation process. In each of the process, there are three or more stages where different types of design decisions are made. In the architectural definition process, the types of design decisions are related to functional, logical and physical elements. In the design/concept definition process, the design decisions are stressed upon design definitions, design characteristics, design enablers and design alternatives. Besides, in the implementation process, the design decisions are now more concrete hence the domain specific design decisions become important. Thus, mechanical, electrical and software design decisions are taken care during the process. 

The authors in \cite{irshorn2007} highlight that the evaluation of design decisions across the initial goals is vital during the product life cycle. There is another significant step that authors stress i.e, documenting the design decisions during the developmental process. They say that it will help to review future decisions.
In the final phase of the product life cycle, a decision analysis process is carried out in order to quantify and asses the current data and suggest further enhancements. This way, it helps to decide on the best ``design solution". There are many decision analysis techniques established to ease the analysis process. I direct the readers to look at \cite{irshorn2007} to view the name, purpose, timeline , results of the analysis process and much more to understand more about the decision analysis techniques.

\section{Software bots}  Bots/Software bots are nothing but an interactive interface for users to avail software services. They can also be termed as artificially intelligent, capable of comprehending commands from users or sometimes smart enough to carry out automated tasks. Collaboration tools like slack and MS teams contain software bots. Developing and hosting simple bots are pretty simple with the latest documentation available in online sources. In fact, Microsoft and Facebook also provide developers with necessary bot frameworks, toolkit and other forms of support. Bots these days are capable of assisting the users with almost every task in their routine\cite{lebeuf_software_2018}. For example --- reminders, support with FAQs, development, presentation and many more. In this regard, the biggest application of a bot in mechatronics is that it aligns the multi disciplinarian team members in terms of design and development process, decision making process etc\cite{storey_botse_nodate}.


Each day, there are novel developments in the research and development of bots. Developers are trying to build such bots that it is capable of supporting the users with natural-language interpretation techniques.  Due to huge advancements in the field of bot development, there is a need to clearly understand what is or is not a bot and hence the authors of \cite{lebeuf_defining_2019} have presented a taxonomy of software bots. The taxonomy is based on environment dimension, intrinsic dimension and interaction dimension. This makes it very clear about the properties, behaviors and the environment where the bots are being operated and designed. This helps the community to better understand, evaluate the existing bots and also how can the future bots be made more innovative and constructive in their design. 


